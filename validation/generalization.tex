Figure \ref{fig:ocl:generalization} presents the invariants
of the \emph{Concept} and \emph{Property} classes
related to \emph{generalizations}:

\begin{itemize}

\item \emph{not\_own\_generalization}:
A \emph{concept} (\ref{ch:concepts}) may not be listed on its own \emph{GeneralizationList},
nor on the \emph{GeneralizationList} of its direct or indirect generalizations.

\item \emph{compatible\_generalizations}:
The \emph{generalizations} of a \emph{concept} must all be compatible between themselves,
that is, no two \emph{generalizations} may have a \emph{property} with the same name
but a different type.

\item \emph{generalization\_compatible\_redefinition}:
A \emph{property} may only be redefined with the same type defined in the \emph{generalizations}.

\item \emph{conflict\_redefinition}:
A \emph{concept} is required to redefine a \emph{property} that 
has been defined by two or more of its \emph{generalizations}
in order to resolve the definition conflict.
That is required only if the \emph{property} has been initialized or derived
in at least one of the \emph{generalizations}.
Otherwise, the redefinition is not required.

\end{itemize}

\begin{figure}
\lstinputlisting[language=ocl_]{ocl/generalization.ocl}
\caption{Generalization Constraints}
\label{fig:ocl:generalization}
\end{figure}
