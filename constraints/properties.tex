Figure \ref{fig:ocl:property} presents the invariants
of the \emph{Property} metaclass:

\begin{itemize}

\item \emph{unique\_property\_name}:
Each \emph{property} must have a unique NAME within its \emph{concept}
(\ref{ch:concepts}).

\item \emph{property\_type\_specified\_or\_inferred}:
Either the \emph{property} explicitly defines a \emph{type}
or it defines an \emph{expression},
from which the type is inferred.
That is required for both regular, slot-based \emph{properties}
(which may provide an \emph{initialization expression})
and \emph{derived properties}
(which may have an \emph{expression} defining the derivation).

\item \emph{property\_type\_assignable\_from\_expression\_type}:
When both a \emph{type} and \emph{expression} are defined for a \emph{property},
the \emph{type} inferred from the \emph{expression} should be assignable to
the declared \emph{type}.
That is required for both regular, slot-based \emph{properties}
(which may provide an \emph{initialization expression})
and \emph{derived properties}
(which may have an \emph{expression} defining the derivation).

\end{itemize}

\begin{figure}
\lstinputlisting[language=ocl_]{ocl/property.ocl}
\caption{Property Constraints}
\label{fig:ocl:property}
\end{figure}
