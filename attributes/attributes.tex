\begin{definition}
In CML, \emph{attributes} are \emph{properties} (\ref{sec:properties})
of \emph{primitive types} (\ref{sec:primitive-types}).
They correspond to the \emph{Attribute} metaclass 
in the ER \cite{er} and UML \cite{uml} metamodels.
\emph{Attributes} serve as a \emph{slot} to hold a value of 
the specified \emph{primitive type}.
An initial value may be specified as an \emph{expression} (\ref{ch:expressions}).
An \emph{attribute}'s value may also be constantly
derived from an \emph{expression} (not only initially),
in which case it is called a \emph{derived attribute} (\ref{sec:derived-attributes}).
While initial values are only set when a \emph{concept} (\ref{ch:concepts})
is instantiated,
the value of \emph{derived attributes} are always evaluated 
from the given \emph{expression},
and they cannot be set any other way.
\end{definition}

\begin{examples}
Figure \ref{fig:ex:attributes} presents some examples of \emph{attributes} declared in CML.
As shown,
the attribute \textbf{a} is a regular attribute definition 
that specifies the \emph{primitive type} (\ref{sec:primitive-types})
of the values that can be held by the \emph{attribute}'s slot.
The attribute \textbf{b} is an example showing how an \emph{attribute}
can be defined with an initial value.
As shown by the attribute \textbf{c}, 
an attribute may be derived from an \emph{expression}
that refers to other \emph{attributes}.
In order to differentiate \emph{attributes} with initial values
from \emph{derived attributes},
a forward slash (``/'') prefixes the name of the latter.
Attributes \textbf{d} and \textbf{e} are examples
where the type of the attribute,
instead of being specified,
is inferred from the given \emph{expression}.
Type inference is possible for both regular, slot-based \emph{attributes}
and \emph{derived attributes} that provide an \emph{expression}.
\end{examples}

\begin{figure}
\verbatimfont{\small}
\lstinputlisting[language=cml]{examples/attributes.cml}
\caption{Examples of Attributes}
\label{fig:ex:attributes}
\end{figure}

\begin{concrete-syntax}
Figure \ref{fig:stx:property} specifies the syntax used
to declare any kind of \emph{property} (\ref{sec:properties}),
including \emph{attributes}.
The NAME of an \emph{attribute} is followed
by a \emph{typeDeclaration} of a \emph{primitive type}
(\ref{sec:primitive-types}).
Optionally, an \emph{expression} (\ref{ch:expressions}) may be specified
in order to set the initial value.
A \emph{derived attribute} must be prefixed with the forward-slash character,
as specified by DERIVED,
in which case the given \emph{expression} defines the value
of the \emph{attribute} at all times.
\end{concrete-syntax}
