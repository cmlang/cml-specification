In CML, \emph{attributes} are \emph{properties} (\ref{ch:properties})
of \emph{primitive types} (\ref{ch:primitive-types}).
They correspond to the \emph{Attribute} metaclass
in the ER \cite{er} metamodel;
in the UML \cite{uml} metamodel,
to the association \emph{attribute} between the metaclass \emph{Class}
and the metaclass \emph{Property}.

\emph{Attributes} serve as a \emph{slot} that holds a value of
the specified \emph{primitive type}.
An initial value may be specified as an \emph{expression} (\ref{ch:expressions}).
Some \emph{attributes}, however, may be continuosly
derive their \emph{value} from an \emph{expression} (not only initially),
in which case they are called \emph{derived attributes} (\ref{ch:derived-attributes}).

While initial values are only set when a \emph{concept} (\ref{ch:concepts})
is instantiated,
the value of \emph{derived attributes} is always evaluated
from the given \emph{expression},
and they cannot be set any other way.
