In CML, all declared \emph{concepts} (\ref{ch:concepts}) are \emph{reference types}.
That means a \emph{property} (\ref{ch:properties}) or any \emph{variable},
whose type is declared or inferred as a concept A, holds a reference
to an instance of concept A; not the actual instance.
This allows other \emph{properties} or \emph{variables}
to also reference to the same instance of A.

Models in CML do not to keep track of the memory used
to store the actual instances.
CML expects the target programming language or technology
to support some kind of reference management,
such as a garbage collector in Java or automatic reference counting in Swift,
or still a database.
CML does not require any particular implementation.

The \emph{path expressions} (\ref{ch:paths})
whose result is of a \emph{concept type} may be used
in \emph{referential expressions} (\ref{ch:referential}),
\emph{type-checking expressions} (\ref{tab:type-checking}),
\emph{type-casting expressions} (\ref{tab:type-casting})
and in \emph{invocations} (\ref{ch:invocations}).
