Figures \ref{fig:ocl:type_a} and \ref{fig:ocl:type_b} define the \emph{isAssignableFrom()}
operation in the \emph{Type} metaclass,
which is used by the \emph{property\_type\_assignable\_from\_expression\_type}
constraint in figure \ref{fig:ocl:property}.
Basically, one of the following conditions must be met for a source \emph{type}
to be assignable to a destination \emph{type}:

\begin{itemize}

\item The source \emph{type} has the same name as the destination \emph{type}.

\item Both types are \emph{numeric} and the destination \emph{type} is wider than the source \emph{type}.
Caveat: Floating-point types (Float and Double) are never assignable to the other \emph{numeric types}
(Byte, Short, Integer, Long), and vice-versa.

\item Both types refer to \emph{concepts} and the destination \emph{concept}
is \emph{generalization} (\ref{ch:generalization}) of the source \emph{concept}.

\end{itemize}

Additionally, one of the following conditions must be met regarding the \emph{type}'s \emph{cardinality}:

\begin{itemize}

\item The cardinality of the source \emph{type} matches the cardinality of the destination \emph{type}.

\item The destination \emph{type} has the \emph{zero-or-one} cardinality and the source \emph{type} has the \emph{one} cardinality.

\item The destination \emph{type} has the \emph{zero-or-more} cardinality and the source \emph{type} has any other cardinality.

\end{itemize}

\begin{figure}
\lstinputlisting[language=ocl_]{ocl/type_a.ocl}
\caption{Auxiliary Methods of The \emph{Type} Metaclass}
\label{fig:ocl:type_a}
\end{figure}

\begin{figure}
\lstinputlisting[language=ocl_]{ocl/type_b.ocl}
\caption{The \emph{isAssignableFrom()} Method of The \emph{Type} Metaclass}
\label{fig:ocl:type_b}
\end{figure}
