Figure \ref{fig:stx:property} specifies the syntax used
to declare a \emph{property}.
The NAME is followed by a \emph{typeDeclaration}
(\ref{ch:primitive-types} and \ref{ch:sequence-types}).
Optionally, an \emph{expression} (\ref{ch:expressions}) may be specified
in order to set the initial value.

\begin{figure}
\verbatimfont{\small}
\lstinputlisting[language=antlr]{grammar/Properties.txt}
\caption{Property Declaration Syntax}
\label{fig:stx:property}
\end{figure}

Figure \ref{fig:meta:property} presents the \emph{Property} metaclass
in an EMOF \cite{mof} class diagram of the CML metamodel,
and figure \ref{fig:ast:property} specifies
the transformation
from the \emph{property} concrete syntax to its abstract syntax.
For each \emph{property} parsed by the compiler,
an instance of the \emph{Property} class will be created,
and its properties will be assigned
according to parsed information:

\begin{itemize}

\item \emph{name}:
assigned with the value of the terminal node NAME.

\item \emph{type}:
if \emph{typeDeclaration} is provided,
\emph{type} is set with the instance of the \emph{Type} class
matching the \emph{typeDeclaration}.

\item \emph{expression}:
if provided,
it contains the instance of the \emph{Expression} class
matching the parsed \emph{expression}.

\end{itemize}

\begin{figure}
\verbatimfont{\small}
\lstinputlisting[language=lsl]{ast/property.lsl}
\caption{Property AST Instantiation}
\label{fig:ast:property}
\end{figure}
