In CML, \emph{referential expressions} are composed by the referential operators,
which only accept as operands the \emph{expressions} (\ref{ch:expressions})
resulting in references (\ref{ch:reference}).
That excludes all \emph{primitive types} (\ref{ch:primitive-types}).
All referential operators are infixed.
There is no associativity between the referential operators.

There are three groups of \emph{referential operators}:
the \emph{referencial-equality operators} (table \ref{tab:referential-equality}),
the \emph{type-checking operators} (table \ref{tab:type-checking})
and the \emph{type-casting operators} (table \ref{tab:type-casting}).

\begin{table}[htbp]
\centering
\begin{tabular}
{ L{1cm} L{3cm} L{4.5cm} l }
\hline
Oper. & Operation & Resulting Type & Example \\
\hline
\verb|===| & Referential Equality & Boolean & \verb|c1 === c2| \\
\verb|!==| & Referencial Inequality & Boolean & \verb|c1 !== c2| \\
\end{tabular}
\caption{Referential-Equality Operators}
\label{tab:referential-equality}
\end{table}

The \emph{referential equality operator} (\verb|is|)
results in the \verb|true| value
if both operands reference the same instance.
The \emph{referential inequality operator} (\verb|is|)
results in \verb|true| if the refecences do not point to the same instance.

\begin{table}[htbp]
\centering
\begin{tabular}
{ L{1cm} L{3cm} L{4.5cm} l }
\hline
Oper. & Operation & Resulting Type & Example \\
\hline
\verb|is| & Type-Checking & Boolean & \verb|member is Task| \\
\verb|isnt| & Negative Type-Checking  & Boolean & \verb|member isnt Task| \\
\end{tabular}
\caption{Type-Checking Operators}
\label{tab:type-checking}
\end{table}

The \emph{type-checking operator} (\verb|is|)
results in the \verb|true| value
if the instance referenced by the first operand
is of the type specified by the second operand,
or it is an \emph{specialization} of such type.
The \emph{nagative type-checking operator} (\verb|is|)
results in \verb|true| if the type does not match.

\begin{table}[htbp]
\centering
\begin{tabular}
{ L{1cm} L{3cm} L{4.5cm} l }
\hline
Oper. & Operation & Resulting Type & Example \\
\hline
\verb|as!| & Uncondicional Type-Casting & Type and cardinality of second operand. & \verb|members as! Task*| \\
\verb|as?| & Condicional Type-Casting & Type of second operand,
only optional (?) or sequential (*) cardinality & \verb|member as? Task?|
\end{tabular}
\caption{Type-Casting Operators}
\label{tab:type-casting}
\end{table}

The \emph{unconditional type-casting operator} (\verb|as!|)
may raise an exception at runtime
if the actual type of the instance is not compatible with the expected type.
The \emph{conditional type-casting operator} (\verb|as?|)
automatically selects the compatible instances,
never raising an expception at runtime.
