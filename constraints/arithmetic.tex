Table \ref{tab:arithmetic-constraints} shows the precende order and associativity of \emph{arithmetic expressions}
in CML.

\begin{table}[H]
\centering
\begin{tabular}
{ l l l }
\hline
Operator & Operation & Associativity \\
\hline
\verb!^! & Exponentiation & Right \\
\verb!*!, \verb!/!, \verb!%! & Multiplication, Division, Modulo & Left  \\
\verb!+!, \verb!-!$^*$ & Addition, Subtraction & Left \\
\multicolumn{3}{l}{\footnotesize{$^*$The addition/subtraction operators
may also be prefixed in the unary form.}}
\end{tabular}
\caption{Arithmetic Operators in Precedence Order}
\label{tab:arithmetic-constraints}
\end{table}

All \emph{arithmetic operators} are infixed,
but the addition (\verb|+|) and subtraction (\verb|-|) operators may also be prefixed when used in the unary form.
The associativity of the arithmetic operators is from left to right,
except for the exponentiation operator (\verb|^|),
where it is from right to left.

A validation error should reported by the compiler if any of the \emph{operands}
of an \emph{arithmetic expression} is inferred to be of a \emph{type}
other than \emph{numeric} (\ref{ch:numeric-types})
or \emph{floating-point} (\ref{ch:floating-point}).
