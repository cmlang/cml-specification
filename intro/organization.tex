The following chapters will specify every element of CML metamodel.
Each chapter starts with a definition, followed by: an example;
the specification of the concrete syntax;
and then presenting the abstract syntax,
and how to transform the concrete syntax into the abstract one.

Chapters may also have sections that specify sub-elements
of the top-level CML metamodel element being described in the chapter level.
Each sub-element is described under its section
using the same definition structure (detailed below)
that is used to define the top-level elements.

\begin{definition}
The definition of each CML metamodel element is stated in plain English
on a paraprah (such as this one)
starting with the ``\textbf{Definition.}'' heading.
If a correspondence exists to an element of
the Entity-Relationship (ER) \cite{er} metamodel,
or to an element of the Unified Modeling Language (UML) \cite{uml} metamodel,
it is provided.
\end{definition}

\begin{examples}
For each metamodel element declaration in CML,
examples are provided on a paraprah (such as this one),
starting with the ``\textbf{Examples.}'' heading.
This type of paragraph refers to a \verb+verbatim+ figure
containing the examples, and describes them as needed.
The examples are provided for illustrative purposes only,
and they are \emph{not} intended to be normative.
They may be excerpts of larger CML source files,
and thus may not be successfully compiled on their own.
\end{examples}

\begin{concrete-syntax}
The concrete syntax of each CML metamodel element is described
on a paragraph (such as this one),
starting with the ``\textbf{Concrete Syntax.}'' heading.
This type of paragraph refers to a \verb+verbatim+ figure,
which contains the actual ANTLR \cite{antlr} grammar
specifying the syntax for the CML metamodel element in question,
and it must be considered normative.
The appendix \ref{apx:concrete-syntax} presents all the grammar rules
in a single listing.
\end{concrete-syntax}

\begin{abstract-syntax}
The abstract syntax of each CML metamodel element is described
on a paragraph (such as this one),
starting with the ``\textbf{Abstract Syntax.}'' heading.
This type of paragraph refers to two types of figure:
the first figure presents a class diagram
with the EMOF \cite{mof}-based metamodel
of the element being described;
the second figure specifies the transformation
from the concrete syntax into instances of the metamodel classes,
which are the nodes of the abstract syntax tree
(the intermediate representation described in section \ref{sec:compiler}).
The notation used to specify the transformations is presented
in the appendix \ref{apx:lsl}.
Both figures must be considered normative.
\end{abstract-syntax}

\begin{constraints}
The constraints of each CML metamodel element are described
on a paragraph (such as this one),
starting with the ``\textbf{Constraints.}'' heading.
This type of paragraph refers to a \verb+verbatim+ figure,
which contains the OCL \cite{ocl} invariants
(and its definitions)
of the CML metamodel element in question,
and it must be considered normative.
Each invariant has a name in the format \verb+inv_name+
so that it can be referred by the compiler's error messages
and users.
The appendix \ref{apx:ocl} presents all the constraint rules
in a single listing.
\end{constraints}

All metamodel elements referred by one of the descriptions defined above
(definitions, examples, etc.)
are emphasized in \emph{italic}.
If the descriptions of a CML metamodel element refer to another CML metamodel element,
the corresponding chapter or section defining the other element
is provided in parenthesis, like so (\ref{sec:org}).

Some sections may not follow the structure defined above.
These normally provide additional semantic information in plain English,
which cannot be described using the notations presented above.
