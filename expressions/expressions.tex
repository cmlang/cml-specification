\begin{definition}
An \emph{expression} may be used in CML to compute values and collections that initialize \emph{properties} or define \emph{derived properties}.
On the UML \cite{uml} metamodel,
it corresponds to an \emph{Expression};
in OCL \cite{ocl}, to \emph{OclExpressionCS}.
The CML \emph{expressions} are designed to provide the same level of
expressivity provided by OCL \emph{expressions},
but the CML syntax varies from OCL, especially for collection operations.
\end{definition}

\begin{examples}
Figure \ref{fig:ex:expressions} presents some examples of CML \emph{expressions}.
As shown, there are different types of expressions:
literals (\ref{sec:literals}),
prefix expressions (\ref{sec:prefix}),
infix expressions (\ref{sec:infix}),
conditional expressions (\ref{sec:conditionals}),
path expressions (\ref{sec:paths})
and queries (\ref{ch:queries}).
\end{examples}

\begin{figure}
\verbatimfont{\small}
\begin{framed}
\verbatiminput{examples/expressions.cml}
\end{framed}
\caption{Expression Examples}
\label{fig:ex:expressions}
\end{figure}
