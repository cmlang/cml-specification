An \emph{abstraction} in CML is a \emph{concept} (\ref{ch:concepts})
that cannot create instances on its own,
but instead serves as a \emph{generalization} (\ref{ch:generalization})
for other \emph{concepts},
which in turn can create their own instances.
Thus, all instances of an \emph{abstraction}
are first instances of its \emph{specializations}.

In CML, an \emph{abstraction} may also define a \emph{derived property} (\ref{ch:properties})
without providing an \emph{expression} (\ref{ch:expressions}) in its definition;
such \emph{properties} are called \emph{abstract properties}.

CML's support for \emph{abstractions} matches UML's \cite{uml},
which allows the declaration of \emph{abstract classes}
by setting the \emph{isAbstract} attribute of a \emph{Class} instance to \emph{true}.
UML also allows the declaration of \emph{abstract attributes} and \emph{abstract operations}.

The original version of the ER \cite{er} metamodel, however,
as a consequence of lacking the \emph{generalization/specialization} relationship,
has not considered the notion of \emph{abstract entities}.
