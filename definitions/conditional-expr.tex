In CML, \emph{conditional expressions} allow alternating between
one or more \emph{expressions} (\ref{ch:expressions})
based on some \emph{condition},
which is an \emph{expression} of \emph{Boolean} type (\ref{ch:boolean}).
The remaining operands of a \emph{conditional expression}
-- the alternating \emph{expressions} --
may be of any type (\ref{ch:types}),
including the \emph{primitive types} (\ref{ch:primitive-types})
and \emph{references} (\ref{ch:references}).

The \emph{conditional expressions} are divided in three categories:

\begin{itemize}
\item unary: only evaluates a single \emph{expression}
if the \emph{condition} evaluates to \verb!true!.
\item binary: results in the evaluation of the first \emph{expression}
if the \emph{condition} evaluates to \verb!true!;
otherwise, it results in the evaluation of the second \emph{expression}.
\item optional: the \emph{condition} is implicit;
it results in the first \emph{expression} if it provides a value;
else, it results in the second \emph{expression} if it provides a value;
otherwise, it results in \emph{none}.
\end{itemize}

The resulting type of a \emph{conditional expression}
is based on the type of its \emph{operands}.
The cardinality is also based on the \emph{operands}.
