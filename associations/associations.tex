\begin{definition}
In CML,
an \emph{association} represents a relation between two \emph{concepts} (\ref{ch:concepts}),
where each \emph{concept} has an \emph{instance} in every tuple that is part of the relation.
When \emph{concepts} have an \emph{association},
its \emph{instances} may be linked in such way that
it is possible to access an \emph{instance} of one \emph{concept}
from an \emph{instance} of the other \emph{concept}.
The UML \cite{uml} metamodel has a metaclass named \emph{Association} that has \emph{Property} instances,
whose \emph{types} are the \emph{Class} instances that are part of the \emph{association}.
In UML, the name of each \emph{Property} instance in the \emph{Association} metaclass
is known as the \emph{role} of the corresponding \emph{Class} in the \emph{association}.
In the CML metamodel, on other hand,
the \emph{Association} metaclass is only needed
when it is necessary to define \emph{bidirectional associations} (\ref{sec:assoc-bidir}).
For \emph{unidirectional associations},
only a \emph{property} is defined in the source \emph{concept},
making its \emph{type} to be the target \emph{concept}.
In the ER \cite{er} metamodel,
each \emph{association} is known as a \emph{relationship set},
and each tuple in this set is called a \emph{relationship}.
Unlike CML and UML,
the tuples in a \emph{relationship set} of any ER model
can be queried directly,
and no notion of \emph{property} is required in an \emph{entity type}
in order to access those \emph{relationships}.
\end{definition}
