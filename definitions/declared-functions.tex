In CML, \emph{functions} may be defined by a \emph{signature}
and a corresponding \emph{expression} (\ref{ch:expressions}).
The \emph{function signature} declares the \emph{parameter}
and the resulting \emph{type} (\ref{ch:types}).
Such a \emph{function} is translated to a target programming language
much like a \emph{derived attribute} (\ref{ch:derived-attributes}),
except that it does not belong to any particular \emph{concept} (\ref{ch:concepts})
and it may be invoked by any \emph{expression}, in any scope.

\emph{Functions} declared in CML are pure functions,
which means they have the following characteristics:

\begin{itemize}
\item they always evaluate to the same result given the same arguments;
\item their evaluation does not cause any side-effects or state mutation.
\end{itemize}

If a \emph{declared function} in its definition invokes other \emph{functions},
then those \emph{functions}
(being them other \emph{declared functions} or \emph{template functions} \ref{ch:template-functions})
must provide the same guarantees listed above in order
for the \emph{declared function} to be considered pure.
