\begin{definition}
An \emph{abstract concept} is one that does not represent specific instances,
but instead serves as a \emph{generalization} (\ref{sec:generalization}) 
for other \emph{concepts},
which in turn represent specific instances.
Thus, all instances of an \emph{abstract concept}
are first instances of its \emph{specializations}.
CML supports tagging a \emph{concept} as \emph{abstract}.
An \emph{abstract concept} in CML may also define a \emph{derived property} (\ref{sec:properties})
wihtout providing an \emph{expression} (\ref{ch:expressions}) in its definition;
such \emph{properties} may also be called \emph{abstract properties}.
CML's support for \emph{abstract concepts} matches UML's \cite{uml},
which allows the declaration of \emph{abstract classes}
-- by setting the \emph{isAbstract} attribute of the \emph{Class} metaclass instance to \emph{true}.
UML also allows the declaration of corresponding \emph{abstract attributes} and \emph{abstract operations}.
The original version of the ER \cite{er} metamodel, however,
as a consequence of lacking the \emph{generalization/specialization} relationship,
has not considered the notion of \emph{abstract entities}.
\end{definition}

\begin{examples}
Figure \ref{fig:ex:abstract} presents an example of an \emph{abstract concept} declared in CML.
As shown, the concept \textbf{Shape} is tagged as \emph{abstract},
and as such no direct instances of \emph{Shape} are ever instantiated.
As an \emph{abstract concept}, \textbf{Shape} can define \emph{abstract properties},
like \textbf{area}, which is just a \emph{derived property} (\ref{sec:properties})
without an \emph{expression} (\ref{ch:expressions}).
An \emph{abstract concept} may also define concrete \emph{properties},
such as \textbf{color} in \textbf{Shape}.
The concept \textbf{Circle} is a \emph{especialization} of \textbf{Shape}
that must redefine the property \textbf{area}
(and provide an \emph{expression})
if it is to be considered a \emph{concrete concept}.
As a \emph{concrete concept},
\textbf{Circle} may have direct instances,
which are in turn instances of \emph{Shape} as well.
\textbf{Circle} may also redefine \emph{concrete properties} of \textbf{Shape},
like \textbf{color},
but the redefinition is not a requirement in this case.
In \textbf{UnitCircle},
we can observe that the redefinition of an \emph{abstract property},
such as \textbf{area},
may be made \emph{concrete};
meaning it does not need to be redefined as a \emph{derived property}.
The converse situation is also allowed in CML,
where a \emph{concrete property} is redefined by as a \emph{derived property},
as illustrated with the property \textbf{radius} in \textbf{UnitCircle}.
\end{examples}

\begin{figure}
\verbatimfont{\small}
\lstinputlisting[language=cml]{examples/abstract.cml}
\caption{Abstract Concept Example}
\label{fig:ex:abstract}
\end{figure}

\begin{concrete-syntax}
Figure \ref{fig:stx:concept} specifies the syntax used
to declare a \emph{concept} (\ref{ch:concepts}) in CML.
It shows that a \emph{concept} may be tagged with the \textbf{abstract} keyword
in order to convey it as an \emph{abstract concept}.
Figure \ref{fig:stx:property} specifies the syntax used 
to declare a \emph{property} (\ref{sec:properties}) in CML.
It shows that a \emph{property} may be prefixed with a forward slash (``/'')
in order to mark it as a \emph{derived property}.
If the optional \textbf{expression} is not provided,
the property is then considered an \emph{abstract property}.
\end{concrete-syntax}

\begin{abstract-syntax}
Figure \ref{fig:meta:concept} presents the \emph{concept} metamodel
in an EMOF \cite{mof} class diagram,
and figure \ref{fig:ast:concept} specifies
the \emph{concept} transformation
from its concrete syntax to its abstract syntax.
There is a \textbf{Boolean} attribute named \textbf{abstract} in the \emph{Concept} class
that determines whether a \emph{concept} is \emph{abstract} or not.
\end{abstract-syntax}
