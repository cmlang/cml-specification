An \emph{expression} in CML is used to compute
\emph{values} (\ref{ch:primitive-types})
or \emph{references} (\ref{ch:concept-types})
to \emph{concept} instances (\ref{ch:concepts}).
They are used to initialize or derive \emph{properties} (\ref{ch:properties}).
On the UML \cite{uml} metamodel,
it corresponds to the \emph{Expression} metaclass;
in OCL \cite{ocl}, to \emph{OclExpressionCS}.

CML \emph{expressions} are designed to provide the same level of
expressivity provided by OCL \emph{expressions},
but the CML syntax varies from OCL's;
sintactically, they differ especially on OCL's \emph{collection operations},
which correspond to \emph{comprehensions} (\ref{ch:comprehensions}) in CML.

CML allows the use of \emph{operators} in \emph{expressions}.
The categories of \emph{operators} in CML are:
\emph{arithmetic operators} (\ref{ch:arithmetic}),
\emph{relational operators} (\ref{ch:relational}),
\emph{logical operators} (\ref{ch:logical}),
\emph{referential operators} (\ref{ch:referential}),
\emph{type-checking operators} (\ref{ch:type-checking}) and
\emph{type-casting operators} (\ref{ch:type-casting}).
Most of the \emph{operators} in CML are infixed,
with just three of them (\verb|not|, \verb|+| and \verb|-|) being used as a prefix.
The use of parenthesis is allowed to estabilish precedence.

Besides the \emph{operators} listed above,
CML also offers the following \emph{expressions}:
\emph{paths} (\ref{ch:paths}),
\emph{invocations} (\ref{ch:invocations}),
\emph{lambdas} (\ref{ch:lambdas})
and \emph{comprehensions} (\ref{ch:comprehensions}).
They are all presented in the following chapters.
