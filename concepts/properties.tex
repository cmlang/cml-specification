\begin{definition}
A \emph{property} in CML may hold values of primitive types,
in which case they correspond to \emph{attributes}
on the ER \cite{er} and UML \cite{uml} metamodels;
or they may hold references (or collections of references)
linking to instances of other \emph{concepts},
in which case they correspond to a \emph{relationship} on the ER metamodel,
and to \emph{associations} on the UML metamodel.
\end{definition}

\begin{examples}
Figure \ref{fig:ex:properties} presents some examples of \emph{properties} declared in CML.
As shown in the examples,
a \emph{property} may be an \emph{attribute} (\ref{ch:attributes})
of a \emph{primitive type} (\ref{sec:primitive-types}),
or represent the role/end of an \emph{association} (\ref{ch:associations}).
\end{examples}

\begin{figure}
\verbatimfont{\small}
\begin{framed}
\verbatiminput{examples/properties.cml}
\end{framed}
\caption{Property Examples}
\label{fig:ex:properties}
\end{figure}

\begin{figure}
\verbatimfont{\small}
\begin{framed}
\verbatiminput{grammar/Properties.txt}
\end{framed}
\caption{Properties Declaration Syntax}
\label{fig:properties-syntax}
\end{figure}
