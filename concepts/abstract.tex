\begin{definition}
An \emph{abstract concept} is one that does not represent specific instances,
but instead serves as a \emph{generalization} (\ref{sec:generalization}) 
for other \emph{concepts},
which in turn represent specific instances.
Thus, all instances of an \emph{abstract concept}
are first instances of its \emph{specializations}.
CML supports tagging a \emph{concept} as \emph{abstract}.
An \emph{abstract concept} in CML may also define a \emph{derived property} (\ref{sec:properties})
wihtout providing an \emph{expression} (\ref{ch:expressions}) in its definition;
such \emph{properties} may also be called \emph{abstract properties}.
CML's support for \emph{abstract concepts} matches UML's \cite{uml},
which allows the declaration of \emph{abstract classes}
-- by setting the \emph{isAbstract} attribute of the \emph{Class} metaclass instance to \emph{true}.
UML also allows the declaration of corresponding \emph{abstract attributes} and \emph{abstract operations}.
The original version of the ER \cite{er} metamodel, however,
as a consequence of lacking the \emph{generalization/specialization} relationship,
has not considered the notion of \emph{abstract entities}.
\end{definition}

\begin{figure}
\verbatimfont{\small}
\lstinputlisting[language=cml]{examples/abstract.cml}
\caption{Abstract Concept Example}
\label{fig:ex:generalization}
\end{figure}
