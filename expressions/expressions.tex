\begin{definition}
An \emph{expression} in CML is used to compute values and collections that initialize \emph{properties} or define \emph{derived properties}.
On the UML \cite{uml} metamodel,
it corresponds to an \emph{Expression};
in OCL \cite{ocl}, to \emph{OclExpressionCS}.
The CML \emph{expressions} are designed to provide the same level of
expressivity provided by OCL \emph{expressions},
but the CML syntax varies from OCL, especially for collection operations.
\end{definition}

\begin{examples}
Figure \ref{fig:ex:expressions} has some examples of CML \emph{expressions}.
As shown, there are different types of expressions:
literals (\ref{sec:literals}),
prefix expressions (\ref{sec:prefix}),
infix expressions (\ref{sec:infix}),
conditional expressions (\ref{sec:conditionals}),
path expressions (\ref{sec:paths})
and queries (\ref{ch:queries}).
\end{examples}

\begin{figure}
\verbatimfont{\small}
\lstinputlisting[language=cml]{examples/expressions.cml}
\caption{Expression Examples}
\label{fig:ex:expressions}
\end{figure}

\begin{concrete-syntax}
Figure \ref{fig:stx:expressions} specifies the syntax of all CML \emph{expressions}. It also lists them in their order of precedence.
Observe that the grammar in figure \ref{fig:stx:expressions} has left
recursions, and thus is ambiguous.
However, ANTLR \cite{antlr} will use the order in which the alternatives
are listed in order to resolve the ambiguity,
and so define the precedence among the operators.
Also, according to ANTLR,
and as required by CML,
all expressions in the grammar are left-to-right associative,
except for the \emph{exponentiation expression},
which is right-to-left associative,
as defined by the \textbf{<assoc=right>} clause.
\end{concrete-syntax}

TODO:
- Escaping quotes in String expressions.

\begin{figure}
\verbatimfont{\small}
\lstinputlisting[language=antlr]{grammar/Expressions.txt}
\caption{Expressions Syntax}
\label{fig:stx:expressions}
\end{figure}
