Listing \ref{lst:stx:property} specifies the syntax used
to declare any kind of \emph{property} (\ref{ch:properties}),
including \emph{attributes}.
The NAME of an \emph{attribute} is followed
by a \emph{typeDeclaration} of a \emph{primitive type}
(\ref{ch:primitive-types}).
Optionally, an \emph{expression} (\ref{ch:expressions}) may be specified
in order to set the initial value.
A \emph{derived attribute} must be prefixed with the forward-slash character,
as specified by DERIVED,
in which case the given \emph{expression} defines the value
of the \emph{attribute} at all times.

Since an \emph{attribute} in CML is just a \emph{property} (\ref{ch:properties})
with \emph{primitive types} (\ref{ch:primitive-types}),
the \emph{property} metaclass in the CML metamodel is used to represent
\emph{attributes}.
Figure \ref{fig:meta:property} presents the \emph{property} metaclass
in an EMOF \cite{mof} class diagram,
and listing \ref{lst:ast:property} specifies
the \emph{property} transformation
from its concrete syntax to its abstract syntax.
