Listing \ref{lst:ex:expressions} has some examples of CML \emph{expressions}.
As shown, there are different types of expressions:
\emph{literal values} (\ref{ch:literals}),
\emph{prefix expressions} (\ref{ch:prefix}),
\emph{infix expressions} (\ref{ch:infix}),
\emph{conditional expressions} (\ref{ch:conditionals}),
\emph{path expressions} (\ref{ch:paths})
and \emph{query expressions} (\ref{ch:queries}).

\begin{code}
\verbatimfont{\small}
\lstinputlisting[language=cml]{examples/expressions.cml}
\caption{Expression Example}
\label{lst:ex:expressions}
\end{code}

