Figure \ref{fig:ex:attributes} presents some examples of \emph{attributes} declared in CML.
As shown,
the attribute \textbf{a} is a regular attribute definition 
that specifies the \emph{primitive type} (\ref{ch:primitive-types})
of the values that can be held by the \emph{attribute}'s slot.
The attribute \textbf{b} is an example showing how an \emph{attribute}
can be defined with an initial value.
As shown by the attribute \textbf{c}, 
an attribute may be derived from an \emph{expression}
that refers to other \emph{attributes}.
In order to differentiate \emph{attributes} with initial values
from \emph{derived attributes},
a forward slash (``/'') prefixes the name of the latter.
Attributes \textbf{d} and \textbf{e} are examples
where the type of the attribute,
instead of being specified,
is inferred from the given \emph{expression}.
Type inference is possible for both regular, slot-based \emph{attributes}
and \emph{derived attributes} that provide an \emph{expression}.

\begin{figure}
\verbatimfont{\small}
\lstinputlisting[language=cml]{examples/attributes.cml}
\caption{Examples of Attributes}
\label{fig:ex:attributes}
\end{figure}
