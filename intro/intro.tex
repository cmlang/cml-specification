The \emph{Conceptual Modeling Language} (CML) is specified in this document.
It allows modeling the information of software systems,
focusing on the structural aspects.
Using CML,
it is possible to represent the information as understood by the system users,
disregarding its physical organization as implemented by the target languages or technologies.

The CML compiler has:
\begin{itemize}
\item as \emph{input},
source files defined using its own conceptual language (as specified in this document),
which provides an abstract syntax similar to (but less comprehensive than) a combination of UML \cite{uml} and OCL \cite{ocl};
\item and, as \emph{output},
any target languages based on extensible templates,
which may be provided by the compiler's base libraries, by third-party libraries, or even by developers.
\end{itemize}

Section \ref{sec:compiler} will provide an overview of the CML compiler's architecture.
Section \ref{sec:org} describes the organization and notation
used in the remainder of this document.
