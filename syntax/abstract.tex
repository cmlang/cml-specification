Figure \ref{fig:stx:concept} specifies the syntax used
to declare a \emph{concept} (\ref{ch:concepts}) in CML.
It shows that a \emph{concept} may be tagged with the \textbf{abstract} keyword
in order to convey it as an \emph{abstract concept}.
Figure \ref{fig:stx:property} specifies the syntax used 
to declare a \emph{property} (\ref{sec:properties}) in CML.
It shows that a \emph{property} may be prefixed with a forward slash (``/'')
in order to mark it as a \emph{derived property}.
If the optional \textbf{expression} is not provided,
the property is then considered an \emph{abstract property}.

Figure \ref{fig:meta:concept} presents the \emph{concept} metamodel
in an EMOF \cite{mof} class diagram,
and figure \ref{fig:ast:concept} specifies
the \emph{concept} transformation
from its concrete syntax to its abstract syntax.
There is a \textbf{Boolean} attribute named \textbf{abstract} in the \emph{Concept} class
that determines whether a \emph{concept} is \emph{abstract} or not.
