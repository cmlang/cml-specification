\begin{definition}
A \emph{concept} (\ref{ch:concepts}) in CML may be generalized by another \emph{concept}.
In other words, a \emph{concept} may be considered a specialization of another \emph{concept}.
Generalized \emph{concepts} have \emph{properties} (\ref{sec:properties})
that apply to a larger set of instances,
while specialized \emph{concepts} have \emph{properties}
that only apply to a subset of those instances.
In the UML \cite{uml} metamodel,
such generalization/specialization relationship between \emph{classes}
is known as \emph{generalization}, which is the name of the metaclass in the UML metamodel.
The original version of the ER \cite{er} metamodel lacked this kind of relationship
between \emph{entity types}.
\end{definition}

\begin{examples}
Figure \ref{fig:ex:generalization} presents some examples of
generalization/specialization relationships declared in CML.
As shown,
a \emph{concept} (\ref{ch:concepts}) may specialize zero or more other \emph{concepts}.
The latter are called the generalizations,
while the former is called the specialization.
A generalization, such as \textbf{Shape},
may define \emph{attributes} (\ref{ch:attributes}),
such as \textbf{color} and \textbf{area},
or also \emph{unidirecional associations} (\ref{sec:assoc-unidir}),
which are \emph{properties} (\ref{sec:properties}) shared among all its specializations.
Some of these \emph{properties} may be redefined by the some of the specializations,
as it is the case with the \emph{area} property,
which is redefined by \textbf{Rectangle}, \textbf{Rhombus} and \textbf{Square}.
Some specializations may also define new \emph{properties},
such as \textbf{width} and \textbf{height} in \textbf{Rectangle},
which characterize only instances of this specialization.
A \emph{concept} may be a specialization of two or more other \emph{concepts},
as seen with \textbf{Square},
which specializes both \textbf{Rectangle} and \textbf{Rhombus},
and thus can redefine \emph{properties} of both generalizations.
If a \emph{property} has been defined by more than one generalization,
then it must be redefined by the specialization
in order to resolve the definition conflict,
which is the case with \textbf{area} in \textbf{Square}. 
If a redefinition suitable for both generalizations is unattainable,
it may be an indication that either the specialization or the generalizations
are unsound from the domain's prospective.
\end{examples}

\begin{figure}
\verbatimfont{\small}
\lstinputlisting[language=cml]{examples/generalization.cml}
\caption{Generalization Examples}
\label{fig:ex:generalization}
\end{figure}

\begin{concrete-syntax}
Figure \ref{fig:stx:concept} specifies the syntax used
to declare a \emph{concept} (\ref{ch:concepts}),
and in turn its generalizations.
A list of NAMEs may be enumerated after the declared \emph{concept}'s NAME,
referring to other \emph{concepts} that this concept is a specialization of.
\end{concrete-syntax}

\begin{abstract-syntax}
Figure \ref{fig:meta:concept} presents the \emph{concept} metamodel
in an EMOF \cite{mof} class diagram,
and figure \ref{fig:ast:concept} specifies
the \emph{concept} transformation
from its concrete syntax to its abstract syntax.
There is a unidirecional association in the \emph{Concept} class
that keeps track of the generalization/specialization relationships,
which is named \emph{generalizations}.
It is an \emph{ordered set} referencing all \emph{concepts}
whose NAMEs were enumerated in the \emph{GeneralizationList}
of the declared \emph{concept}.
\end{abstract-syntax}

\begin{constraints}
Figure \ref{fig:ocl:generalization} presents the invariants
of the \emph{Concept} and \emph{Property} classes
related to \emph{generalizations}:

\begin{itemize}

\item \emph{not\_own\_generalization}:
A \emph{concept} (\ref{ch:concepts}) may not be listed on its own \emph{GeneralizationList},
nor on the \emph{GeneralizationList} of its direct or indirect generalizations.

\item \emph{compatible\_generalizations}:
The \emph{generalizations} of a \emph{concept} must all be compatible between themselves,
that is, no two \emph{generalizations} may have a \emph{property} with the same name
but different types.

\item \emph{redefinition\_compatible\_with\_generalizations}:
A \emph{property} may only be redefined with the same type defined in the \emph{generalizations}.

\item \emph{conflict\_redefinition}:
A \emph{concept} is required to redefine a \emph{property} that 
has been defined by two or more of its \emph{generalizations}
in order to resolve the definition conflict.

\end{itemize}
\end{constraints}

\begin{figure}
\lstinputlisting[language=ocl_]{ocl/generalization.ocl}
\caption{Generalization Constraints}
\label{fig:ocl:generalization}
\end{figure}
