CML allows the use of \emph{lambda expressions}
as an argument of the \emph{invocation} (\ref{ch:invocations})
of a \emph{function} (\ref{ch:built-in-functions}).
CML provides some \emph{built-in functions} in the \verb|cml_base| module
that accept a \emph{lambda expression} as an argument.

\emph{Lambda expressions} may be seen as an inline function definition.
Just like \emph{functions},
\emph{lambda expressions} may have a number of \emph{parameters}
that are used in its inner expression.

CML also allows a parameterless \emph{lambda expression}
to have an undeclared, implicit parameter,
which serves as its scope.
This implicit parameter in the \emph{lambda expression}
is inferred from the \emph{function}
declaring the \emph{lambda} as one of its parameters.
\emph{Functions} may declare a \emph{lambda parameter}
using a \emph{function types} (\ref{ch:function-types}).

Additionally, \emph{lambda expressions} in CML are considered
pure functions (just like \emph{declared functions} \ref{ch:declared-functions})
given that the \emph{invocations} performed by the inner expressions
is also pure \emph{functions} or \emph{lambdas}.
