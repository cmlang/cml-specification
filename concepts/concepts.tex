A \emph{concept} in the CML metamodel is used
to represent complex information
that has a coherent, cohesive meaning in a domain.
On the ER \cite{er} metamodel,
it corresponds to an \emph{entity};
on the UML \cite{uml} metamodel,
to a \emph{class}.
The CML \emph{concept} differs, however, from the UML \emph{class},
because it only contains \emph{properties},
while the UML \emph{class} may also have \emph{operations}.

\begin{figure}
\verbatimfont{\small}
\begin{framed}
\verbatiminput{grammar/Concepts.txt}
\end{framed}
\caption{Concept Declaration Syntax}
\label{fig:concept-syntax}
\end{figure}

\section{Properties}\label{sec:properties}

\begin{figure}
\verbatimfont{\small}
\begin{framed}
\verbatiminput{grammar/Properties.txt}
\end{framed}
\caption{Properties Declaration Syntax}
\label{fig:properties-syntax}
\end{figure}

\section{Inheritance}\label{sec:inheritance}
