Figure \ref{fig:ex:abstract} presents an example of an \emph{abstract concept} declared in CML.
As shown, the concept \textbf{Shape} is tagged as \emph{abstract},
and as such no direct instances of \emph{Shape} are ever instantiated.
As an \emph{abstract concept}, \textbf{Shape} can define \emph{abstract properties},
like \textbf{area}, which is just a \emph{derived property} (\ref{sec:properties})
without an \emph{expression} (\ref{ch:expressions}).
An \emph{abstract concept} may also define concrete \emph{properties},
such as \textbf{color} in \textbf{Shape}.
The concept \textbf{Circle} is a \emph{especialization} of \textbf{Shape}
that must redefine the property \textbf{area}
(and provide an \emph{expression})
if it is to be considered a \emph{concrete concept}.
As a \emph{concrete concept},
\textbf{Circle} may have direct instances,
which are in turn instances of \emph{Shape} as well.
\textbf{Circle} may also redefine \emph{concrete properties} of \textbf{Shape},
like \textbf{color},
but the redefinition is not a requirement in this case.
In \textbf{UnitCircle},
we can observe that the redefinition of an \emph{abstract property},
such as \textbf{area},
may be made \emph{concrete};
meaning it does not need to be redefined as a \emph{derived property}.
The converse situation is also allowed in CML,
where a \emph{concrete property} is redefined by as a \emph{derived property},
as illustrated with the property \textbf{radius} in \textbf{UnitCircle}.

\begin{figure}
\verbatimfont{\small}
\lstinputlisting[language=cml]{examples/abstract.cml}
\caption{Abstract Concept Example}
\label{fig:ex:abstract}
\end{figure}

