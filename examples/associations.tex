Figure \ref{fig:ex:associations} presents some examples of \emph{associations} declared in CML.
The concept \textbf{Vehicle} contains the property \textbf{driver},
which may optionally refer to an instance of \textbf{Employee},
meaning that a \textbf{driver} may or may not be assigned to a single \textbf{Vehicle}.
The concept \textbf{Vehicle} also has the property \textbf{owner},
which always refers to an instance of \textbf{Organization},
meaning that an \textbf{owner} must always be assigned to each instance of \textbf{Vehicle}. 
Similarly,
the concept \textbf{Employee} has the property \textbf{employer},
which must always be assigned to an instance of \textbf{Organization}.

Just below the declaration of \textbf{Organization},
we observe an association named \textbf{Employment},
which enumerates two \emph{properties}:
the first is \textbf{employer} from the concept \textbf{Employee};
the second is \textbf{employees} from the concept \textbf{Organization}.
What this \emph{association} implies is a correspondence between these two properties.
Every time a reference to an instance of \textbf{Organization} is assigned to
the slot \textbf{employer} of an instance of \textbf{Employee},
a reference to this same instance of \textbf{Employee} must be assigned to
the slot \textbf{employees} of the \textbf{Organization} instance.
However,
since the \emph{type} of \textbf{employees}
in the concept \textbf{Organization}
is a sequence (\ref{ch:sequence-types}) of \textbf{Employee} instances,
the reference to the instance of \textbf{Employee} will actually be appended to the sequence
being held by the slot \textbf{employees} of the concept \textbf{Organization},
and maintained along with the other \textbf{Employee} instances already found in the sequence.
Thus, the association \textbf{Employment} actually characterizes a \emph{bidirectional association}.

The association \textbf{VehicleOwnership} is another example of a \emph{bidirectional association};
in this case,
between \textbf{Vehicle}'s \textbf{owner} property and \textbf{Organization}'s \textbf{fleet} property.
It can be noticed, though, 
in this second \emph{bidirectional association},
that the \emph{types} of the \emph{properties} are declared along with their names;
such a \emph{type} declaration,
in the \emph{association} declaration,
is optional in CML,
but must match the original \emph{property} declaration under the \emph{concept} declaration,
if present.

The \textbf{driver} property in the concept \textbf{Vehicle} is a different case,
since this \emph{property} does not participate in any \emph{association} declaration
in figure \ref{fig:ex:associations}.
That's because there is no corresponding \emph{property} in the concept \textbf{Employee}
representing the other end of the \emph{association}.
As such, the property \textbf{driver} is representing the source end of a \emph{unidirectional association}.

The property \textbf{drivers} in the concept \textbf{Organization}
is \emph{derived association} (\ref{ch:derived-associations}).

\begin{figure}
\verbatimfont{\small}
\lstinputlisting[language=cml]{examples/associations.cml}
\caption{Association Example}
\label{fig:ex:associations}
\end{figure}

