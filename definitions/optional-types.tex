A \emph{referential type} (\ref{ch:referential-types})
with cardinality zero-or-one is called an \emph{optional type}.
It is declared optional by the question-mark (\verb|?|) suffix
after the \emph{concept}'s name in a type declaration.

In CML, \emph{primitive types} (\ref{ch:primitive-types})
and \emph{function types} (\ref{ch:function-types})
cannot be declared as \emph{optional}.

\emph{Expressions} (\ref{ch:expressions})
of the \emph{optional type} may combined
with \emph{expressions} of any other \emph{referential type}
using the \emph{referential operators} (\ref{ch:referential}).
The \emph{empty()} and \emph{present()} functions may also be used
to determine whether an \emph{expression} is empty or not.
And the \emph{none} constant may be assigned to a \emph{property}
of the \emph{optional type}.
