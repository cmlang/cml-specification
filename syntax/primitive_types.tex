Figure \ref{fig:stx:type} specifies the syntax used
to declare any kind of \emph{type},
including \emph{primitive types}.
The NAME of the \emph{type} may be any of the \emph{primitive types}
defined in the column named \emph{CML}
of the tables \ref{tab:core-primitive-types} and \ref{tab:additional-primitive-types}.
Optionally, cardinality may also be specified
for a \emph{primitive type}.
The `*' cardinality suffix allows zero or more values to be stored
in a property as a sequence type.
The `?' cardinality suffix allows a single value to be stored, or none.
If no cardinality is specified,
a value must be assigned to the \emph{attribute}
when its \emph{concept} is instantiated.

\begin{figure}
\verbatimfont{\small}
\lstinputlisting[language=antlr_tiny]{grammar/Types.txt}
\caption{Type Declaration Syntax}
\label{fig:stx:type}
\end{figure}

Figure \ref{fig:meta:association} presents the \emph{Type} metaclass
in an EMOF \cite{mof} class diagram of the CML metamodel,
and figure \ref{fig:ast:type} specifies
the transformation
from the \emph{type} concrete syntax to its abstract syntax.

\begin{figure}
\verbatimfont{\small}
\lstinputlisting[language=lsl]{ast/type.lsl}
\caption{Type AST Instantiation}
\label{fig:ast:type}
\end{figure}
