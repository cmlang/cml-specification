A \emph{primitive type} in CML is one of the pre-defined \emph{data types}
supported by the language,
as shown in tables \ref{tab:core-primitive-types} and \ref{tab:additional-primitive-types}.

In the ER \cite{er} metamodel,
a \emph{data type} is formally defined as a \emph{set} of \emph{values}
that can be held by an \emph{attribute} (\ref{ch:attributes}).
The original ER paper \cite{er} states that,
for each \emph{value set} (i.e. \emph{data type}),
there is a \emph{predicate} that can be used to test
whether a \emph{value} belongs to the \emph{set}.
In CML, instead,
\emph{literal expressions} are syntactically defined for each \emph{primitive type},
so that the \emph{type} can be inferred from the \emph{literal expression}.

On the original ER paper,
it is also said that \emph{values} in a \emph{value set}
may be equivalent to \emph{values} in another \emph{value set}.
In CML, also,
\emph{literal expressions} of the \emph{Integer} type may be equivalent 
to \emph{literal expressions} of the \emph{Decimal},
and so with other \emph{numeric types}.
This allows \emph{expressions} of a \emph{primitive type}
to be promoted to \emph{expressions} of another \emph{primitive type}
in order to allow \emph{type inference} of composite \emph{expressions},
such as \emph{infix expressions} (\ref{sec:infix}).

In the UML \cite{uml} metamodel,
there is a specific metaclass named \emph{PrimitiveType},
which matches to the same notion in CML.

\begin{table}[h]
\centering
\begin{tabular}
{l l l l l p{2cm} }
\hline
CML & Java & C\# & C++ & Python & TypeScript (JavaScript) \\
\hline
String & String & string & std::wstring & str & string \\
\multicolumn{6}{p{13cm}}{\footnotesize{16-bit Unicode character sequences.}} \\
\\
Boolean & boolean & bool & bool & bool & boolean \\
\multicolumn{6}{p{13cm}}{\footnotesize{Only values are the literal expressions: \textbf{true}, \textbf{false}.}}  \\
\\
Integer & int & int & int32\_t & int & number  \\
\multicolumn{6}{p{13cm}}{\footnotesize{32-bit signed two's complement integer.}}  \\
\\
Decimal* & BigDecimal & decimal & decimal128 & Decimal & number \\
\multicolumn{6}{p{13cm}}{\footnotesize{Arbitrary precision,
fixed-point,
or decimal floating-point, 
depending on the target language.}} \\
\\
\multicolumn{6}{p{13cm}}{*The specification of Decimal type varies by target programming language.
Compared to the binary floating-point types (Float and Double),
the Decimal type is better suited for monetary calculations
at a performance cost.}
\end{tabular}
\caption{Core Primitive Types in CML.}
\label{tab:core-primitive-types}
\end{table}

\begin{table}[h]
\centering
\begin{tabular}
{l l l l l p{2cm} p{3.5cm} }
\hline
CML & Java & C\# & C++ & Python & TypeScript (JavaScript) & Specification \\
\hline
Byte & byte & byte & int8\_t & int & number & 8-bit signed two's complement integer \\
Short & short & short & int16\_t & int & number & 16-bit signed two's complement integer \\
Long & long & long & int64\_t & long & number & 64-bit signed two's complement integer \\
Float & float & float & float* & float & number & 32-bit IEEE 754 binary floating point \\
Double & double & double & double* & float & number & 64-bit IEEE 754 binary floating point \\
\\
\multicolumn{7}{p{12cm}}{*C++ floating point types may vary by hardware and compiler}
\end{tabular}
\caption{Additional Primitive Types in CML.}
\label{tab:additional-primitive-types}
\end{table}

\begin{examples}
Figure \ref{fig:ex:primitive-types} presents examples
of \emph{atributes} declared with \emph{primitive types} in CML.
Each example corresponds to one of the \emph{primitive types} 
supported by the language,
as shown in tables \ref{tab:core-primitive-types} and \ref{tab:additional-primitive-types}.
The \emph{target constructors} (\ref{sec:constructors})
of CML's base module will translate the primitive types to Java, C\#, C/C++,
Python, and TypeScript (JavaScript),
according to the mapping shown in the tables.

\end{examples}

\begin{figure}
\verbatimfont{\small}
\lstinputlisting[language=cml]{examples/primitive_types.cml}
\caption{Example of \emph{Primitive Types}}
\label{fig:ex:primitive-types}
\end{figure}

\begin{concrete-syntax}
Figure \ref{fig:stx:type} specifies the syntax used
to declare any kind of \emph{type},
including \emph{primitive types}.
The NAME of the \emph{type} may be any of the \emph{primitive types}
defined in the column named \emph{CML}
of the tables \ref{tab:core-primitive-types} and \ref{tab:additional-primitive-types}.
Optionally, cardinality may also be specified
for a \emph{primitive type}.
The `*' cardinality suffix allows zero or more values to be stored
in a property as a collection type (\ref{sec:collection-types}).
The `?' cardinality suffix allows a single value to be stored, or none.
If no cardinality is specified,
a value must be assigned to the \emph{attribute}
when its \emph{concept} is instantiated.
\end{concrete-syntax}

\begin{figure}
\verbatimfont{\small}
\lstinputlisting[language=antlr]{grammar/Types.txt}
\caption{Type Declaration Syntax}
\label{fig:stx:type}
\end{figure}

\begin{abstract-syntax}
Figure \ref{fig:meta:property} presents the \emph{Type} metaclass
in an EMOF \cite{mof} class diagram of the CML metamodel,
and figure \ref{fig:ast:type} specifies
the transformation
from the \emph{type} concrete syntax to its abstract syntax.
\end{abstract-syntax}

\begin{figure}
\verbatimfont{\small}
\lstinputlisting[language=lsl]{ast/type.lsl}
\caption{Type AST Instantiation}
\label{fig:ast:type}
\end{figure}

\begin{constraints}
Figures \ref{fig:ocl:type_a} and \ref{fig:ocl:type_b} define the \emph{isAssignableFrom()}
operation in the \emph{Type} metaclass,
which is used by the \emph{property\_type\_assignable\_from\_expression\_type}
constraint in figure \ref{fig:ocl:property}.
Basically, one of the following conditions must be met for a source \emph{type}
to be assignable to a destination \emph{type}:

\begin{itemize}

\item The source \emph{type} has the same name as the destination \emph{type}.

\item Both types are \emph{numeric} and the destination \emph{type} is wider than the source \emph{type}.
Caveat: Floating-point types (Float and Double) are never assignable to the other \emph{numeric types}
(Byte, Short, Integer, Long), and vice-versa.

\item Both types refer to \emph{concepts} and the destination \emph{concept}
is \emph{generalization} (\ref{ch:generalization}) of the source \emph{concept}.

\end{itemize}

Additionally, one of the following conditions must be met regarding the \emph{type}'s \emph{cardinality}:

\begin{itemize}

\item The cardinality of the source \emph{type} matches the cardinality of the destination \emph{type}.

\item The destination \emph{type} has the \emph{zero-or-one} cardinality and the source \emph{type} has the \emph{one} cardinality.

\item The destination \emph{type} has the \emph{zero-or-more} cardinality and the source \emph{type} has any other cardinality.

\end{itemize}

\end{constraints}

\begin{figure}
\lstinputlisting[language=ocl_]{ocl/type_a.ocl}
\caption{Auxiliary Methods of The \emph{Type} Metaclass}
\label{fig:ocl:type_a}
\end{figure}

\begin{figure}
\lstinputlisting[language=ocl_]{ocl/type_b.ocl}
\caption{The \emph{isAssignableFrom()} Method of The \emph{Type} Metaclass}
\label{fig:ocl:type_b}
\end{figure}
