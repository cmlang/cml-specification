In CML, \emph{referential expressions} are composed by the referential operators,
which only accept as operands the \emph{expressions} (\ref{ch:expressions})
resulting in references (\ref{ch:reference}).
That excludes all \emph{primitive types} (\ref{ch:primitive-types}).
All referential operators are infixed.
There is no associativity between the referential operators.

The table \ref{tab:referential-equality} presents
the \emph{referencial-equality operators}.

\begin{table}[htbp]
\centering
\begin{tabular}
{ L{1cm} L{3cm} L{4.5cm} l }
\hline
Oper. & Operation & Resulting Type & Example \\
\hline
\verb|===| & Referential Equality & Boolean & \verb|c1 === c2| \\
\verb|!==| & Referencial Inequality & Boolean & \verb|c1 !== c2| \\
\end{tabular}
\caption{Referential-Equality Operators}
\label{tab:referential-equality}
\end{table}

The \emph{referential equality operator} (\verb|is|)
results in the \verb|true| value
if both operands reference the same instance.
The \emph{referential inequality operator} (\verb|is|)
results in \verb|true| if the refecences do not point to the same instance.
